\documentclass{article}
%---------------------------------------------------------------------------------------------------------------------------------
% Packages
\usepackage[margin=1in,includefoot]{geometry}
\usepackage{enumitem}
\usepackage{caption}
\usepackage{titlesec}
\usepackage{amsmath}
\usepackage{empheq}
\usepackage{imakeidx}
\usepackage[utf8]{inputenc}
\usepackage{color}
\usepackage[acronym,toc,style=tree]{glossaries}
\usepackage[automake]{glossaries-extra}
\usepackage{graphicx}
\usepackage{hyperref}
%---------------------------------------------------------------------------------------------------------------------------------
\graphicspath{{./Images/}}
%---------------------------------------------------------------------------------------------------------------------------------
% TOC Hyperlink Setup
\hypersetup{
    colorlinks,
    citecolor=black,
    filecolor=black,
    linkcolor=blue,
    urlcolor=black
}
%---------------------------------------------------------------------------------------------------------------------------------
% Sets automatic numbering system in \paragraph
\setcounter{secnumdepth}{4}
\titleformat{\paragraph}
{\normalfont\normalsize\bfseries}{\theparagraph}{1em}{}
\titlespacing*{\paragraph}
{0pt}{3.25ex plus 1ex minus .2ex}{1.5ex plus .2ex}
%---------------------------------------------------------------------------------------------------------------------------------
% Custom Box for Mathematical Formulae
\definecolor{myblue}{rgb}{.8, .8, 1}
\newlength\mytemplen
\newsavebox\mytempbox
\makeatletter
\newcommand\mybluebox{%
    \@ifnextchar[%]
       {\@mybluebox}%
       {\@mybluebox[0pt]}}
\def\@mybluebox[#1]{%
    \@ifnextchar[%]
       {\@@mybluebox[#1]}%
       {\@@mybluebox[#1][0pt]}}
\def\@@mybluebox[#1][#2]#3{
    \sbox\mytempbox{#3}%
    \mytemplen\ht\mytempbox
    \advance\mytemplen #1\relax
    \ht\mytempbox\mytemplen
    \mytemplen\dp\mytempbox
    \advance\mytemplen #2\relax
    \dp\mytempbox\mytemplen
    \colorbox{myblue}{\hspace{1em}\usebox{\mytempbox}\hspace{1em}}}
\makeatother
%---------------------------------------------------------------------------------------------------------------------------------
% Allows for the creation of the glossary
\makeglossaries
%---------------------------------------------------------------------------------------------------------------------------------
% Glossary Terms

% -- Instantaneous Voltage
\newglossaryentry{InstantaneousTime}
{
	name=Instantaneous Time,
	description={An instant in time}
}

\newglossaryentry{TimePeriod}
{
	name=Time Period,
	description={A period between two points in time. Usually describes a full period in a sinusoidal wave}
}

\newglossaryentry{InitialTime}
{
	name=Initial Time,
	description={Describes the initial time that a period in time is taken}
}

\newglossaryentry{InstantaneousVoltage}
{
	name=Instantaneous Voltage,
	description={The voltage across two reference points at time t}
}

\newglossaryentry{AverageVoltage}
{
	name=Average Voltage,
	description={The average voltage across two reference points throughout a period of time T}
}

\newglossaryentry{InstantaneousCurrent}
{
	name=Instantaneous Current,
	description={The current going through a reference point at time t}
}

\newglossaryentry{AverageCurrent}
{
	name=Average Current,
	description={The average current going through a reference point throughout a period of time T}
}

\newglossaryentry{InstantaneousPower}
{
	name=Instantaneous Power,
	description={The power an element is consuming or producing at time t}
}

\newglossaryentry{AveragePower}
{
	name=Average Power,
	description={The average power an element is consuming or producing throughout a period of time T}
}

\newglossaryentry{Work}
{
	name=Work,
	description={The power a system is able to output over a period in time}
}

\newglossaryentry{Inductance}
{
	name=Inductance,
	description={A measurement of the phenomenon that change in current is resisted in an inductor due to the establishment of a magnetic field}
}

\newglossaryentry{Capacitance}
{
	name=Capacitance,
	description={A measurement of the phenomenon that change in voltage is resisted in a capacitor dut to the establishment of an electric field}
}
%---------------------------------------------------------------------------------------------------------------------------------
% Variables

% --- Time
\newacronym{instTime}{$\pmb{t}$}{Instantaneous Time (s)}
\newacronym{timePeriod}{$\pmb{T}$}{Time Period ($\Delta$s)}
\newacronym{initTime}{$\pmb{t_o}$}{Initial Time (s)}

% --- Voltage
\newacronym{instVoltage}{$\pmb{v(t)}$}{Instantaneous Voltage (V)}
\newacronym{initVoltage}{$\pmb{v(t_o)}$}{Initial Voltage at Time $t_o$ (V)}
\newacronym{avgVoltage}{$\pmb{V}$}{Average Voltage (V)}
\newacronym{rmsVoltage}{$\pmb{V_{RMS}}$}{RMS Voltage (V)}

% --- Current
\newacronym{instCurrent}{$\pmb{i(t)}$}{Instantaneous Current (A)}
\newacronym{initCurrent}{$\pmb{i(t_o)}$}{Initial Current at Time $t_o$ (A)}
\newacronym{avgCurrent}{$\pmb{I}$}{Average Current (A)}
\newacronym{rmsCurrent}{$\pmb{I_{RMS}}$}{RMS Current (A)}

% --- Power
\newacronym{instPower}{$\pmb{p(t)}$}{Instantaneous Power (W)}
\newacronym{avgPower}{$\pmb{P}$}{Average Power (W)}
\newacronym{pf}{$\pmb{pf}$}{Power Factor}

% --- Work
\newacronym{work}{$\pmb{W}$}{Work (J)}

% --- Source
\newacronym{sVoltage}{$\pmb{v_s(t)}$}{Voltage Emitted from Source (V)}
\newacronym{sCurrent}{$\pmb{i_s(t)}$}{Current Emitted from/to a Source (A)}

% --- Resistors
\newacronym{resistance}{$\pmb{R}$}{Resistance ($\Omega$)}
\newacronym{resAvgPower}{$\pmb{P_R}$}{Average Power Consumed by a Resistor (W)}

% --- Inductance
\newacronym{inductance}{$\pmb{L}$}{Inductance (H)}
\newacronym{indVoltage}{$\pmb{v_L(t)}$}{Instantaneous Voltage Across an Inductor (V)}
\newacronym{indAvgVoltage}{$\pmb{V_L}$}{Average Voltage Across and Inductor (V)}
\newacronym{indCurrent}{$\pmb{i_L(t)}$}{Instantaneous Current Through an Inductor (A)}
\newacronym{indInitCurrent}{$\pmb{i_L(t_o)}$}{Initial Current Through an Inductor (A)}
\newacronym{indAvgCurrent}{$\pmb{I_L}$}{Average Current Through an Inductor (A)}
\newacronym{indInstPower}{$\pmb{p_L(t)}$}{Instantaneous Power an Inductor Produces/Consumes (W)}
\newacronym{indAvgPower}{$\pmb{P_L}$}{Average Power an Inductor Produces/Consumes (W)}
\newacronym{indInstWork}{$\pmb{w_L(t)}$}{Instantaneous Work an Inductor Does at Time t (J)}
\newacronym{indAvgWork}{$\pmb{W_L}$}{Work an Inductor Does over a Period of Time (J)}

% --- Capacitance
\newacronym{capacitance}{$\pmb{C}$}{Capacitance (F)}
\newacronym{capVoltage}{$\pmb{v_C(t)}$}{Instantaneous Voltage Across a Capacitor (V)}
\newacronym{capInitVoltage}{$\pmb{v_C(t_o)}$}{Initial Voltage Across a Capacitor (V)}
\newacronym{capAvgVoltage}{$\pmb{V_C}$}{Average Voltage Across a Capacitor (V)}
\newacronym{capCurrent}{$\pmb{i_C(t)}$}{Instantaneous Current Through a Capacitor (A)}
\newacronym{capAvgCurrent}{$\pmb{I_C}$}{Average Current Through a Capacitor(A)}
\newacronym{capInstPower}{$\pmb{p_C(t)}$}{Instantaneous Power a Capacitor Produces/Consumes (W)}
\newacronym{capAvgPower}{$\pmb{P_C}$}{Average Power a Capacitor Produces/Consumes (W)}
\newacronym{capInstWork}{$\pmb{w_L(t)}$}{Instantaneous Work a Capacitor Does at Time t (J)}
\newacronym{capAvgWork}{$\pmb{W_C}$}{Work a Capacitor Does over a Period of Time (J)}
%---------------------------------------------------------------------------------------------------------------------------------
% Index Stuff
\makeindex[columns=1]
%---------------------------------------------------------------------------------------------------------------------------------
\begin{document}
\pagenumbering{gobble}
%---------------------------------------------------------------------------------------------------------------------------------
% Title Page
\title{Power Electronics Reference}
\author{Angelino Lefevers}
\maketitle
\thispagestyle{empty}
\cleardoublepage
%---------------------------------------------------------------------------------------------------------------------------------
% Table of Contents
\tableofcontents
\thispagestyle{empty}
\cleardoublepage
%---------------------------------------------------------------------------------------------------------------------------------
% Introduction
\section*{Introduction}
\addcontentsline{toc}{section}{Introduction}
Just throwing together some concepts and examples for later reference
\cleardoublepage
%---------------------------------------------------------------------------------------------------------------------------------
% --- --- --- --- --- --- --- ---  General Equations --- --- --- --- --- --- --- --- 
\section{General Equations}
\pagenumbering{arabic}
\setcounter{page}{1}
%---------------------------------------------------------------------------------------------------------------------------------
% Power and Energy
\begin{minipage}{\linewidth}
\subsection{Power and Energy}

% Instantaneous Power
\subsubsection{Instantaneous Power}
This value is a positive real number if the element is supplying power, or is a negative real number if the element is consuming power.

\begin{empheq}[box={\mybluebox[5pt]}]{equation*}
p(t)=v(t)i(t)\quad(W)
\end{empheq}

\begin{itemize}	%*********************************************************************
\item \acrshort{instPower} := \acrlong{instPower} \index{\acrlong{instPower}}					% p(t)
\item \acrshort{instVoltage} := \acrlong{instVoltage} \index{\acrlong{instVoltage}}				% v(t)
\item \acrshort{instCurrent} := \acrlong{instCurrent} \index{\acrlong{instCurrent}}				% i(t)
\end{itemize}	%*********************************************************************
\end{minipage}

% Average Power
\begin{minipage}{\linewidth}
\subsubsection{Average Power}

\begin{empheq}[box={\mybluebox[5pt]}]{equation*}
P=\frac{1}{T}\int_{t_o}^{t_o+T}p(t)dt\quad(W)
\end{empheq}

\begin{empheq}[box={\mybluebox[5pt]}]{equation*}
P=\frac{1}{T}\int_{t_o}^{t_o+T}v(t)i(t)dt\quad(W)
\end{empheq}

\begin{empheq}[box={\mybluebox[5pt]}]{equation*}
P=\frac{W}{T}\quad(W)
\end{empheq}

\begin{itemize}	%*********************************************************************
\item \acrshort{avgPower} := \acrlong{avgPower} \index{\acrlong{avgPower}}					% P
\item \acrshort{instPower} := \acrlong{instPower} \index{\acrlong{instPower}}					% p(t)
\item \acrshort{instVoltage} := \acrlong{instVoltage} \index{\acrlong{instVoltage}}				% v(t)
\item \acrshort{instCurrent} := \acrlong{instCurrent} \index{\acrlong{instCurrent}}				% i(t)
\item \acrshort{timePeriod} := \acrlong{timePeriod}											% T
\item \acrshort{initTime} := \acrlong{initTime}												% to
\item \acrshort{work} := \acrlong{work} \index{\acrlong{work}}								% W
\end{itemize}	%*********************************************************************
\end{minipage}

% Energy
\begin{minipage}{\linewidth}
\subsubsection{Energy}

\begin{empheq}[box={\mybluebox[5pt]}]{equation*}
W=\int_{t_1}^{t_2}p(t)dt\quad(J)
\end{empheq}

\begin{itemize}	%*********************************************************************
\item \acrshort{work} := \acrlong{work} \index{\acrlong{work}}								% W
\item \acrshort{instPower} := \acrlong{instPower} \index{\acrlong{instPower}}					% p(t)
\end{itemize}	%*********************************************************************
\end{minipage}

% Inductors
\begin{minipage}{\linewidth}
\subsection{Inductors}

% --- Current
\subsubsection{Current/Voltage Relationship}

\begin{empheq}[box={\mybluebox[5pt]}]{equation*}
i_L(t_o+T) = \frac{1}{L}\int_{t_o}^{t_o+T}v_L(t)dt+i_L(t_o)
\end{empheq}

\begin{itemize}	%*********************************************************************
\item \acrshort{indCurrent} := \acrlong{indCurrent} \index{\acrlong{indCurrent}}				% iL(t)
\item \acrshort{indVoltage} := \acrlong{indVoltage} \index{\acrlong{indVoltage}}				% vL(t)
\item \acrshort{indInitCurrent} := \acrlong{indInitCurrent} \index{\acrlong{indInitCurrent}}		% iL(to)
\item \acrshort{inductance} := \acrlong{inductance} \index{\acrlong{inductance}}				% L
\item \acrshort{initTime} := \acrlong{initTime}												% to
\item \acrshort{timePeriod} := \acrlong{timePeriod}											% T
\end{itemize}	%*********************************************************************

\begin{empheq}[box={\mybluebox[5pt]}]{equation*}
V_L = \frac{1}{T}\int_{t_o}^{t_o+T}v_L(t)dt
\end{empheq}

NOTE: When a sinusoidal current is applied through the inductor, the average voltage is zero ($V_L = 0$) for each period.

\begin{itemize}	%*********************************************************************
\item \acrshort{indAvgVoltage} := \acrlong{indAvgVoltage} \index{\acrlong{indAvgVoltage}}		% VL
\item \acrshort{indVoltage} := \acrlong{indVoltage} \index{\acrlong{indVoltage}}				% vL(t)
\item \acrshort{initTime} := \acrlong{initTime}												% to
\item \acrshort{timePeriod} := \acrlong{timePeriod}											% T
\end{itemize}	%*********************************************************************
\end{minipage}

% --- Power and Energy
\begin{minipage}{\linewidth}
\subsubsection{Power and Energy}

\begin{empheq}[box={\mybluebox[5pt]}]{equation*}
w_L(t) = \frac{1}{2}Li_L^2(t)
\end{empheq}

\begin{itemize}	%*********************************************************************
\item \acrshort{indInstWork} := \acrlong{indInstWork} \index{\acrlong{indInstWork}}			% wL(t)
\item \acrshort{indCurrent} := \acrlong{indCurrent} \index{\acrlong{indCurrent}}				% iL(t)
\item \acrshort{inductance} := \acrlong{inductance} \index{\acrlong{inductance}}				% L
\end{itemize}	%*********************************************************************

\begin{empheq}[box={\mybluebox[5pt]}]{equation*}
P_L = \frac{1}{T}\int_{t_o}^{t_o+T}p_L(t)dt
\end{empheq}

NOTE: The average power produced/consumed by the inductor is zero ($P_L = 0$) for each period when a periodic sinusoidal current is applied through the inductor.

\begin{itemize}	%*********************************************************************
\item \acrshort{indAvgPower} := \acrlong{indAvgPower} \index{\acrlong{indAvgPower}}		% PL
\item \acrshort{indInstPower} := \acrlong{indInstPower} \index{\acrlong{indInstPower}}		% p(t)L
\item \acrshort{initTime} := \acrlong{initTime}												% to
\item \acrshort{timePeriod} := \acrlong{timePeriod}											% T
\end{itemize}	%*********************************************************************
\end{minipage}
\clearpage

% Capacitors
\begin{minipage}{\linewidth}
\subsection{Capacitors}

% --- Current
\subsubsection{Current/Voltage Relationship}

\begin{empheq}[box={\mybluebox[5pt]}]{equation*}
v_C(t_o+T) = \frac{1}{C}\int_{t_o}^{t_o+T}i_C(t)dt + v_C(t_0)
\end{empheq}

\begin{itemize}	%*********************************************************************
\item \acrshort{capVoltage} := \acrlong{capVoltage} \index{\acrlong{capVoltage}}				% vC(t)
\item \acrshort{capCurrent} := \acrlong{capCurrent} \index{\acrlong{capCurrent}}				% iC(t)
\item \acrshort{capInitVoltage} := \acrlong{capInitVoltage} \index{\acrlong{capInitVoltage}}		% vC(to)
\item \acrshort{capacitance} := \acrlong{capacitance} \index{\acrlong{capacitance}}				% C
\item \acrshort{initTime} := \acrlong{initTime}												% to
\item \acrshort{timePeriod} := \acrlong{timePeriod}											% T
\end{itemize}	%*********************************************************************

\begin{empheq}[box={\mybluebox[5pt]}]{equation*}
I_C = \frac{1}{T}\int_{t_o}^{t_o+T}i_C(t)dt
\end{empheq}

NOTE: When a sinusoidal voltage is applied through a capacitor, the average current is zero ($I_C = 0$) for each period.

\begin{itemize}	%*********************************************************************
\item \acrshort{capAvgCurrent} := \acrlong{capAvgCurrent} \index{\acrlong{capAvgCurrent}}	% IC
\item \acrshort{capCurrent} := \acrlong{capCurrent} \index{\acrlong{capCurrent}}				% iC(t)
\item \acrshort{initTime} := \acrlong{initTime}												% to
\item \acrshort{timePeriod} := \acrlong{timePeriod}											% T
\end{itemize}	%*********************************************************************
\end{minipage}

% --- Power and Energy
\begin{minipage}{\linewidth}
\subsubsection{Power and Energy}

\begin{empheq}[box={\mybluebox[5pt]}]{equation*}
w_C(t) = \frac{1}{2}Cv_C^2(t)
\end{empheq}

\begin{itemize}	%*********************************************************************
\item \acrshort{capInstWork} := \acrlong{capInstWork} \index{\acrlong{capInstWork}}			% wC(t)
\item \acrshort{capVoltage} := \acrlong{capVoltage} \index{\acrlong{capVoltage}}				% vC(t)
\item \acrshort{capacitance} := \acrlong{capacitance} \index{\acrlong{capacitance}}				% C
\end{itemize}	%*********************************************************************

\begin{empheq}[box={\mybluebox[5pt]}]{equation*}
P_C = \frac{1}{T}\int_{t_o}^{t_o+T}p_L(t)dt
\end{empheq}

NOTE: The average power produced/consumed by a capacitor is zero ($P_C = 0$) for each period when a periodic sinusoidal current is applied through the inductor.

\begin{itemize}	%*********************************************************************
\item \acrshort{capAvgPower} := \acrlong{capAvgPower} \index{\acrlong{capAvgPower}}		% PC
\item \acrshort{capInstPower} := \acrlong{capInstPower} \index{\acrlong{capInstPower}}		% pC(t)
\item \acrshort{initTime} := \acrlong{initTime}												% to
\item \acrshort{timePeriod} := \acrlong{timePeriod}											% T
\end{itemize}	%*********************************************************************
\end{minipage}
\clearpage

% RMS Values
\begin{minipage}{\linewidth}
\subsection{Root Mean Square (RMS)}
The Root Mean Square (RMS) magnitude of voltages and currents identify the effective power applied to a circuit from a source. This is usually compared to the power consumed by each element in the circuit to calculate the efficiency of a circuit.

\begin{empheq}[box={\mybluebox[5pt]}]{equation*}
V_{RMS}=\sqrt{\frac{1}{T}\int_{t_o}^{t_o+T}v_S^2(t)dt}
\end{empheq}

\begin{empheq}[box={\mybluebox[5pt]}]{equation*}
V_{RMS}=\sqrt{V_{1,RMS}^2 + V_{2,RMS}^2 + V_{3,RMS}^2 + ... }
\end{empheq}

\begin{empheq}[box={\mybluebox[5pt]}]{equation*}
I_{RMS}=\sqrt{\frac{1}{T}\int_{t_o}^{t_o+T}i_S^2(t)dt}
\end{empheq}

\begin{empheq}[box={\mybluebox[5pt]}]{equation*}
I_{RMS}=\sqrt{I_{1,RMS}^2 + I_{2,RMS}^2 + I_{3,RMS}^2 + ... }
\end{empheq}

\begin{itemize}	%*********************************************************************
\item \acrshort{rmsVoltage} := \acrlong{rmsVoltage} \index{\acrlong{rmsVoltage}}				% Vrms
\item \acrshort{rmsCurrent} := \acrlong{rmsCurrent} \index{\acrlong{rmsCurrent}}				% Irms
\item \acrshort{sVoltage} := \acrlong{sVoltage} \index{\acrlong{sVoltage}}						% vS(t)
\item \acrshort{sCurrent} := \acrlong{sCurrent} \index{\acrlong{sCurrent}}						% iS(t)
\item \acrshort{initTime} := \acrlong{initTime}												% to
\item \acrshort{timePeriod} := \acrlong{timePeriod}											% T
\end{itemize}	%*********************************************************************
\end{minipage}
\clearpage

% Apparent Power
\begin{minipage}{\linewidth}
\subsection{Apparent Power}
Apparent power relates the true and reactive powers of a circuit. The true power is what is usually calculated with the fundamental formula ($P=IV$), and is measured in watts. The reactive power is represented by the variable Q, and describes how inductors and capacitors change the voltage and current of a circuit without actually drawing power. Reactive power is usually returned back to the source and is measured in units Volt-Amp-Reactive (VAR). Apparent power relates these two phenomenons in a trigonometric fashion with true power (P) and reactive power (Q) on the x and y axis, and apparent power (S) on the hypoteneuse. The angle used to relate each magnitude to eachother is the impedence phase angle, with apparent power equaling the tangent of the reactive power divided by true power ($S = tan(\frac{Q}{P})$).

\begin{empheq}[box={\mybluebox[5pt]}]{equation*}
S = V_{RMS}I_{RMS}
\end{empheq}

\begin{itemize}	%*********************************************************************
\item \acrshort{rmsVoltage} := \acrlong{rmsVoltage} \index{\acrlong{rmsVoltage}}				% Vrms
\item \acrshort{rmsCurrent} := \acrlong{rmsCurrent} \index{\acrlong{rmsCurrent}}				% Irms
\end{itemize}	%*********************************************************************

% Power Factor
\subsection{Power Factor}

\begin{empheq}[box={\mybluebox[5pt]}]{equation*}
pf = \frac{P}{S} = \frac{P}{V_{RMS}I_{RMS}}
\end{empheq}

\begin{itemize}	%*********************************************************************
\item \acrshort{pf} := \acrlong{pf} \index{\acrlong{pf}}										% pf
\item \acrshort{avgPower} := True Power (W)\index{True Power (W)}							% P
\item \acrshort{rmsVoltage} := \acrlong{rmsVoltage} \index{\acrlong{rmsVoltage}}				% Vrms
\item \acrshort{rmsCurrent} := \acrlong{rmsCurrent} \index{\acrlong{rmsCurrent}}				% Irms
\end{itemize}	%*********************************************************************

\end{minipage}
\clearpage
%---------------------------------------------------------------------------------------------------------------------------------
% --- --- --- --- --- --- --- --- Sinusoidal AC Circuits --- --- --- --- --- --- --- --- 
\begin{minipage}{\linewidth}
\section{Sinusoidal AC Circuits}

% Definitions
\subsection{Definitions}

\subsubsection{Voltage and Current Definitions for any and all Elements}

\begin{empheq}[box={\mybluebox[5pt]}]{equation*}
\begin{split}
v(t) = V_mcos(\omega t + \Theta)  \\
i(t) = I_mcos(\omega t + \phi)
\end{split}
\end{empheq}

\subsubsection{Real Power}

\begin{empheq}[box={\mybluebox[5pt]}]{equation*}
p(t) = (\frac{V_mI_m}{2})[cos(2\omega t + \Theta + \phi) + cos(\Theta - \phi)]
\end{empheq}

\begin{empheq}[box={\mybluebox[5pt]}]{equation*}
P = (\frac{V_mI_m}{2})cos(\Theta - \phi)
\end{empheq}

\subsubsection{Reactive Power}

\begin{empheq}[box={\mybluebox[5pt]}]{equation*}
Q = V_{RMS}I_{RMS}sin(\Theta-\phi)
\end{empheq}

\subsubsection{Apparent Power}

\begin{empheq}[box={\mybluebox[5pt]}]{equation*}
S = P + jQ = (V_{RMS})(I_{RMS})^*
\end{empheq}

\end{minipage}
\clearpage
%---------------------------------------------------------------------------------------------------------------------------------
% Appendices, Indexes, and Glossaries
\pagenumbering{roman}
\setcounter{page}{1}

%\addcontentsline{toc}{section}{Variables}
\printglossary[type=\acronymtype,nonumberlist,title={Variables}]
\clearpage
%\addcontentsline{toc}{section}{Glossary}
\glsaddall
\printglossary[nonumberlist]
\clearpage
%---------------------------------------------------------------------------------------------------------------------------------
\printindex
\clearpage
%---------------------------------------------------------------------------------------------------------------------------------
\end{document}
%---------------------------------------------------------------------------------------------------------------------------------