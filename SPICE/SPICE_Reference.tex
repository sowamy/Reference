\documentclass{article}
%----------------------------------------------------------------
% Packages
\usepackage[margin=0.5in]{geometry}
\usepackage{caption}
\usepackage{hyperref}
\usepackage{url}
%----------------------------------------------------------------
% Caption Types
\DeclareCaptionType{Table}
\DeclareCaptionType{Script}
%----------------------------------------------------------------
% Table of contents link setup
\hypersetup{
    colorlinks,
    citecolor=black,
    filecolor=black,
    linkcolor=blue,
    urlcolor=black
}
%----------------------------------------------------------------
% Glossary Terms

%----------------------------------------------------------------
\begin{document}
\fboxsep=8pt\relax
\fboxrule=2pt\relax
\pagenumbering{gobble}
%----------------------------------------------------------------
% Title Page
\title{SPICE Reference}
\author{Angelino Lefevers}
\maketitle
\thispagestyle{empty}
\cleardoublepage
%----------------------------------------------------------------
% Table of Contents
\tableofcontents
\thispagestyle{empty}
\cleardoublepage
%----------------------------------------------------------------
% Introduction
\section*{Introduction}
\addcontentsline{toc}{section}{Introduction}
This reference will compile a large amount of data and information of SPICE circuit sofware, a variety of distributions of SPICE simulation software, and a variety of distributions of SPICE graphical software. This reference will focus on the compilation and organization of this data and information. This reference will also act as an index of many resources which have been used to find this information, and will be notated in the bibliography of this reference. This reference is not purely written by the author notated on the title page and directly uses text and images from the resources noted in the bibliography section at the end.
\cleardoublepage 
%----------------------------------------------------------------
% --- --- General Information about SPICE -- -- 
\section{General Information about SPICE}
This section presents general information about the history and general use of SPICE.
\pagenumbering{arabic}
\setcounter{page}{1}
%----------------------------------------------------------------
% What is SPICE?
\subsection{What is SPICE?}
SPICE is a powerful general purpose analog circuit simulator that is used to verify circuit designs and to predict the circuit behavior. This is of particular importance for integrated circuits. It was for this reason that SPICE was originally developed at the Electronics Research Laboratory of the University of California, Berkeley (1975), as its name implies: \\ 
\begin{center}
\textbf{S}imulation \textbf{P}rogram for \textbf{I}ntegrated \textbf{C}ircuits \textbf{E}mphasis. 
\end{center}
\cleardoublepage
%----------------------------------------------------------------
% --- --- SPICE Overview --- --- 
\section{Overview}
This section begins to present the things that SPICE are able to do and how it's able to do those things. 
%----------------------------------------------------------------
% Types of Circuit Analysis
\subsection{Types of Circuit Analysis}
SPICE allows for a variety of types of simulation to be conducted on a circuit. Each types of analysis allows an engineer to verify and analyze the operation of a physical circuit. These analysis also help an engineer to identify inconsistencies between ideal circuit calculations and the operation of empirical circuitry.
\begin{itemize}
\item DC
\item AC Small Signal 
\item Transient
\item Pole Zero
\item Noise
\item Sensitivity
\item Distortion
\item Fourier
\item Monte Carlo
\end{itemize}
%----------------------------------------------------------------
% SPICE Scaling Factors
\subsection{Scaling Factors}
SPICE uses a set of characters to scale the magnitudes which describe element and circuit characteristics. When a magnitude in spice is appended with one of these characters, it is scaled accordingly.
\begin{center}
\begin{tabular}{||c|c|c||}
\hline
Suffix & Name & Factor \\
\hline \hline
T & Tera & $10^{12}$ \\
\hline
G & Giga & $10^9$ \\
\hline
Meg & Mega & $10^6$ \\
\hline
K & Kilo & $10^3$ \\
\hline
mil & Mil & $25.4x10^{-6}$ \\
\hline
m & milli & $10^{-3}$ \\
\hline
u & micro & $10^{-6}$ \\
\hline
n & nano & $10^{-9}$ \\ 
\hline
p & pico & $10^{-12}$ \\
\hline
f & femto & $10^{-15}$ \\
\hline
\end{tabular}
\end{center}
\captionof{Table}{SPICE Scaling Factors}
%----------------------------------------------------------------
% Standard SPICE Components
\subsection{Standard Components}
SPICE simulates circuits by using combinations of standard sets of components which are featured within the SPICE simulating software. SPICE circuitry can also use custom components created by 3rd parties to reflect the operation of virtually any piece of hardware, whether it is realizable physically or not. Each component in a SPICE circuit is represented by a name, the first letter of which describes the type of component. The types of components that can be implemented in a SPICE circuit are listed below with the preliminary letter which corresponds to each use.  
\begin{center}
\begin{tabular}{||c|c||}
\hline
Letter & Element Description \\
\hline \hline
A & XSPICE Code Model \\
\hline
B & Behavioral Source \\
\hline
C & Capacitor \\
\hline
D & Diode \\
\hline
E & Voltage-Controlled Voltage Source \\
\hline
F & Current-Controlled Current Source \\
\hline
G & Voltage-Controlled Current Source \\
\hline
H & Current-Controlled Voltage Source \\
\hline
I & Current Source \\
\hline
J & Junction Field Effect Transistor (JFET) \\
\hline
K & Coupled (Mutual) Inductors \\
\hline
L & Inductor \\
\hline
M & Metal Oxide Field Effect Transistor (MOSFET) \\
\hline
N & Numerical Device for GSS \\
\hline
O & Lossy Transmission Line \\
\hline
P & Coupled Multiconductor Line (CPL) \\
\hline
Q & Bipolar Junction Transistor (BJT) \\
\hline
R & Resistor \\
\hline
S & Switch (Voltage Controlled) \\
\hline
T & Lossy Transmission Line \\
\hline
U & Uniformly Distributed RC Line \\
\hline
V & Voltage Source \\
\hline
W & Switch (Current Controlled) \\
\hline
X & Subcircuit\\
\hline
Y & Single Lossy Transmission Line (TXL) \\
\hline
Z & Metal Semiconductor Field Effect Transistor (MESFET) \\
\hline
\end{tabular}
\end{center}
\captionof{Table}{SPICE Elements}
\cleardoublepage
%----------------------------------------------------------------
% SPICE Script Layout
\subsection{SPICE Script Layout}
Below in Script \ref{LAYOUT} lies the pseudocode of a SPICE program. This illustrates how the general SPICE program is partitioned and where each functional element lies.
\begin{center}
\fbox{
  \parbox{8cm}{
    TITLE 							\\
    ELEMENTS						\\
    .								\\
    .								\\
    CONTROL STATEMENTS				\\
    OUTPUT STATEMENTS				\\
    .END 
  }
}
\captionof{Script}{SPICE Program Layout (Pseudocode)\label{LAYOUT}}
\end{center}
\cleardoublepage
%----------------------------------------------------------------
% --- --- SPICE General Script --- ---
\section{General Script}
%----------------------------------------------------------------
% SPICE Comments
\subsection{Comments}
No script is complete without the ability to add comments into the original text. This allows further description of elements, functions, variables, and the ideas that go into implemented functionality. SPICE has two ways to insert comments into its code. The first way is illustrated in \ref{COMMENT1}, this method comments the entire line of text after the \textbf{*} symbol. The other way to comment in SPICE is to comment the remainder of a line, this is done with the \textbf{\$} symbol before the commented code.

\begin{center}
\fbox{
  \parbox{8cm}{
		* \textit{comment}
  }
}
\captionof{Script}{Line Commenting\label{COMMENT1}}
\end{center}

\begin{center}
\fbox{
  \parbox{8cm}{
		R2 0 1 100 \$ \textit{comment} \\
		R3 1 2 200 \$ \textit{comment}
  }
}
\captionof{Script}{End of Line Comment\label{TITLE1}}
\end{center}
%----------------------------------------------------------------
% Circuit Title
\subsection{Circuit Title}
Each circuit description must begin with a title. The general form a title is described is illustrated below in Script \ref{TITLE1}. The title can also be written as is without any sort of tag as displayed in Script \ref{TITLE2}
% This allows for the script to be accessed externally by other files, and gives some description of the operation of the circuit.
\begin{center}
\fbox{
  \parbox{8cm}{
		.TITLE \textit{title} 
  }
}
\captionof{Script}{Declaring a Title with Tag TITLE\label{TITLE1}}
\end{center}

\begin{center}
\fbox{
  \parbox{8cm}{
		\textit{title}
  }
}
\captionof{Script}{Short Way to Declare a Title (Without Tag)\label{TITLE2}}
\end{center}

% Arguments
\begin{itemize}
\item \textit{title} := Title of circuit
\end{itemize}
%----------------------------------------------------------------
% .END
\subsection{.END}
Every circuit description must be finished by an \textbf{.END} tag, which signifies the end of the script. This end tag is illustrated below in Script \ref{END}, and is as simple as it looks both there and in Script \ref{END}. There are no arguments that go into this tag and is to be used as is.

\begin{center}
\fbox{
  \parbox{8cm}{
		.END
  }
}
\captionof{Script}{End Script Tag\label{END}}
\end{center}

\cleardoublepage
%----------------------------------------------------------------
% --- --- SPICE Circuit Elements --- ---
\section{Standard Circuit Elements} 
\cleardoublepage
%----------------------------------------------------------------
% Appendices, Indexes, and Glossaries
\pagenumbering{roman}
\setcounter{page}{1}
%----------------------------------------------------------------
% Citations (So that all references show in the bibliography)
\nocite{PennEngineering}	
\nocite{ngspiceManual27}
%----------------------------------------------------------------
% Bibliography
\addcontentsline{toc}{section}{References}
\bibliographystyle{plain}
\bibliography{SPICE_Bibliography}
%----------------------------------------------------------------
\end{document}